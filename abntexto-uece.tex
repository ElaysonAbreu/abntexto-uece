%! abntexto-uece.cls
%! Public Domain Software 2025
%! 2025-08-27
%! 1.0
%! Elayson Abreu
%! abntexto.classe@gmail.com

% !TeX TS-program = lualatex
\documentclass{abntexto-uece}

\usepackage[brazil,shorthands=off]{babel}
\usepackage[cmyk]{xcolor}
\usepackage{lipsum}
\usepackage[
    style       = abnt,
    maxbibnames = 100,
    extrayear   = true,
]{biblatex} \addbibresource{abntexto-uece.bib}
\usepackage{unicode-math}
\usepackage[colorlinks,linktoc=page]{hyperref}
\usepackage{microtype}
\usepackage{fontspec}
\usepackage{listings}
\usepackage{lua-widow-control} \clubpenalty=10000\widowpenalty=10000

\makeatletter

% VERBATIM
% ================================================

\def\adef#1{\catcode`#1=13 \begingroup \lccode`\~=`#1\lowercase{\endgroup\def~}}
\def\setverb{\def\do##1{\catcode`##1=12}\dospecials\space}
\def\verbchar#1{%
    \ifx\savedttchar\undefined\else \catcode\savedttchar=\savedttcharc \fi
    \chardef\savedttchar=`#1\relax
    \chardef\savedttcharc=\catcode`#1\relax
    \adef{#1}{\leavevmode\hbox\bgroup\setverbchar\readverb}%
    \bgroup\lccode`\~=`#1\lowercase{\egroup\def\readverb ##1~}{##1\egroup}%
    \catcode`#1=13
}
\def\setverbchar{\setverb
    \adef{ }{\ }%
    \adef{\{}{\bgroup\normalcolor\char`\{}%
    \adef{\}}{\char`\}\egroup}%
    \adef{[}{\bgroup\normalcolor\char`[}%
    \adef{]}{\char`]\egroup}%
    \ttfamily
}

\edef\bslash{\csstring\\}
\def\mytarget#1{\vbox to0pt{\kern-12pt \hypertarget{#1}{}\vss}}
\def\1{`}\def\2{``}
{\catcode`\`=13 \AtBeginDocument{%
    \verbchar{`}%
    \def\^`{\bgroup \docpoint}%
    \def\docpoint#1`{\egroup\leavevmode\edef\tmp{\csstring#1}%
        \expandafter\parsecn \string#1\relax
        \ifcsname \cn s/^\tmp\endcsname \else
            \mytarget{\cn s/^\tmp}\csgdef{\cn s/^\tmp}{}%
        \fi
        \hyperlink{\cn s/\tmp}{\color{blue}\ttfamily\if\cn c\bslash\fi\tmp}%
    }%
    \def\parsecn #1#2\relax {\edef\cn{\if\bslash#1c\else n\fi}}%
    \def\`{\bgroup \mainpoint}%
    \def\mainpoint #1`{\egroup\leavevmode\edef\tmp{\csstring#1}%
        \expandafter\parsecn \string#1\relax
        \ifcsname \cn s/\tmp\endcsname \moremainpoints \else
            \mytarget{\cn s/\tmp}\csgdef{\cn s/\tmp}{}%
        \fi
        \ifcsname \cn s/^\tmp\endcsname
            \hyperlink{\cn s/^\tmp}{\ttfamily\if\cn c\bslash\fi\tmp}\else
            {\color{red}\ttfamily\string#1}%
        \fi
    }%
    \def\moremainpoints{\errmessage{Second main documentation point \if\cn c\bslash\fi\tmp}}
}}

\let\lchevron=<
{\catcode`<=13 \AtBeginDocument{\catcode`<=13
    \def<#1>{\ifhmode\else\expandafter\hbox\fi\bgroup\normalcolor\ttfamily
             $\color{gray}\langle$\/{\rmfamily\itshape #1\/}$\color{gray}\rangle$\egroup}}%
}

\def\type#1{\NoCaseChange{\protect\typeA{#1}}}
\def\typeA#1{\bgroup\normalshape\ttfamily{\escapechar=-1\relax\expandafter}\detokenize{#1}\egroup}
% Precisamos de \m@ne aqui, \m@ne=-1.
\pdfstringdefDisableCommands{%
    \def\type#1{\bgroup\escapechar\m@ne\expandafter\egroup\detokenize{#1}}%
    \let\escapechar=\relax
    \let\m@ne=\relax
}

\definecolor{P}{RGB}{242, 121, 0}   % Primitivas.
\definecolor{C}{RGB}{23, 230, 23}   % Comentários.
\definecolor{B}{RGB}{245, 250, 247} % Fundo.

\lstset{
    language        = [latex]tex,
    basicstyle      = \abntsmall\singlesp\ttfamily,
    texcsstyle      = *\color{P}, % O listings não permite ":"
    commentstyle    = \color{C},  % no nome dessas cores. Bug.
    backgroundcolor = \color{B},
    frame           = leftline,
    rulecolor       = \color{lightgray},
    columns         = fullflexible,
    breaklines      = false,
    keepspaces      = true,
    aboveskip       = 0pt plus1.2ex minus.67ex,
    belowskip       = 0pt plus1.2ex minus.67ex,
    abovecaptionskip = 0pt,
    belowcaptionskip = 0pt,
%    tabsize         = 1, % Sem efeito, porque o documento usa Espaço na tabulação.
}

% #1 = linewidth
% #2 = border
% #3 = padding
% #4 = padding-top
% #5 = padding-right
% #6 = padding-bottom
% #7 = padding-left
% #8 = margin-left
% #9 = margin-right
\def\mylstset#1#2#3#4#5#6#7#8#9{\lstset{
    linewidth          = \dimexpr#1 -#2 -#3 -#5\relax,
    framerule          = \dimexpr#2\relax,
    framesep           = \dimexpr#3\relax,
    xleftmargin        = \dimexpr#2 +#3 +#7 +#8\relax,
    xrightmargin       = \dimexpr#9\relax,
    framextopmargin    = \dimexpr#4\relax,
    framexrightmargin  = \dimexpr#5\relax,
    framexbottommargin = \dimexpr#6\relax,
    framexleftmargin   = \dimexpr#7\relax,
}}
\mylstset{\linewidth}{1pt}{0pt}{0pt}{0pt}{0pt}{0pt}{0pt}{0pt}

\lstnewenvironment{lst}[1][\placepos]{%
    \let\medskipamount=\smallskipamount
    \place[#1]\vbox\bgroup
}{%
    \egroup\endplace
}
\lstnewenvironment{longlst}[1][\placepos]{%
    \begingroup
    \processplacearg{#1}%
    \parskip=0pt \parindent=0pt
    \initplace
    \printlegendbox
    \resetORIlabel \ignorespaces
}{%
    \unskip
    \savedplacewidth=\hsize
    \printsrcbox
    \finishplace
    \resetplace \gresetORIlabel
    \endgroup
}

\def\hookXP{}
\lstnewenvironment{xp}[1][]{%
    \lstset{
        frame           = leftline,
        basicstyle      = \normalsize\ttfamily\singlesp\vskip-\baselineskip,
        texcsstyle      = \normalcolor,
        commentstyle    = \normalcolor,
        backgroundcolor = \color{white},
        gobble          = 4,
    }%
    \mylstset{\linewidth}{0pt}{0pt}{0pt}{0pt}{0pt}{\parindent}{0pt}{0pt}%
    \expandafter\lstset \expandafter{\hookXP,#1}%
    \endgraf\null
}{}

% ETC.
% ================================================

\setmainfont{XITS}[
    UprightFont    = *-Regular,
    BoldFont       = *-Bold,
    ItalicFont     = *-Italic,
    BoldItalicFont = *-BoldItalic,
    Extension      = .otf
]
\setmathfont{XITSMath-Regular.otf}
\setmonofont{InconsolataN}[
    UprightFont = *-Regular,
    BoldFont    = *-Bold,
    Extension   = .otf
]

\definelegendplace{code}{Código}{loc}
%\definelegendplace{output}{Saída}{loo}
%\definelegendplace{scr}{\emph{Screenshot}}{loscr}

\let\over=\@@over
\AtBeginDocument{\def\refname{Referências}} % Removido \^ da definição.
\def\hooksection{}
\AtBeginDocument{\pretocmd\appendix{\clearpage}{}{}}
\pretocmd\annex{\clearpage}{}{}
\def\etex{\leavevmode\hbox{$\varepsilon$-\TeX}}
\def\pdflatex{\leavevmode\hbox{PDF\LaTeX}}
\def\xelatex{\leavevmode\hbox{Xe\LaTeX}}
\def\lualatex{\leavevmode\hbox{Lua\LaTeX}}
\def\me{Elaboração própria.}
\def\bibfont{\raggedright\interlinepenalty=10000\singlesp\bibitemsep=\baselineskip}
%\appto\textual{
%    \def\@evenhead{\abntsmall \firstmarks0\hfil\thepage}
%    \def\@oddhead {\abntsmall \botmarks1\hfil\thepage}
%    \def\sectionmark   #1{\marks0{Seção \thesection: \unexpanded{#1}}}
%    \def\subsectionmark#1{\marks1{Subseção \thesubsection: \unexpanded{#1}}}
%    \preto\@oddhead {\lower4pt\rlap{\vrule width\hsize height.4pt}}
%    \preto\@evenhead{\lower4pt\rlap{\vrule width\hsize height.4pt}}
%}
%\patchcmd{\eletroniclayout}{false}{true}{}{}
\let\onesidelayout=\eletroniclayout
\let\twosidelayout=\eletroniclayout
\let\Metadados=\relax
\hypersetup{
    pdfauthor   = Elayson Abreu,
    pdftitle    = Classe ABNTexto-UECE: trabalhos
                  conforme as normas da UECE,
    pdfsubject  = Manual da classe ABNTexto-UECE,
    pdfcreator  = LaTeX with abntexto-uece,
    pdfkeywords = ABNTexto-UECE; LaTeX; Classe; ABNT.
}
\makeatother

\begin{document}

\leavevmode\vskip1.9in
\begingroup \centering \LARGE\sffamily\bfseries
    Classe ABNTexto-UECE\vskip2em \normalfont\large
    Elayson Abreu\vskip-3pt
    \href{mailto:abntexto.classe@gmail.com}%
                {abntexto.classe@gmail.com}\vskip2ex
    \today
\vskip2in\endgroup

\noindent\hfil {\Large\ttfamily 1.0}
\newpage

\nonum\notoc\section{Sumário}
\maketoc
\newpage

\section{Introdução}

Esta classe é destinada aos estudantes da UECE como auxílio na criação de trabalhos acadêmicos: TCCs, dissertações e teses. Ela foi baseada na classe `abntexto.cls` disponível no CTAN: \url{https://ctan.org/pkg/abntexto} e no Guia de Normalização de Trabalhos Acadêmicos da UECE: \url{https://www.uece.br/biblioteca/wp-content/uploads/sites/27/2024/09/GUIA-UECE-2024-Atualizado-1.pdf}, 5ª edição, V1, 2024.

Exemplo de uso desta classe disponível no arquivo `abntexto-uece-exemplo.tex` (Código~\ref{cd:ex}):

\legend{code}{Exemplo de uso} \label{cd:ex}
\src\me
\begin{longlst}
    % !TeX TS-program = lualatex
    \documentclass{abntexto-uece}

    \usepackage{hyperref}
    \usepackage[english,brazil]{babel}
    \usepackage{fontspec}
    \usepackage{unicode-math}

    \Autor{Fulano de Tal}
    \Universidade{Universidade Estadual do Ceará}
    \SiglaDaInstituicao{SIGLA}

    % Estes três são mutuamente exclusivos.
    %\Programa{Programa de Pós-Graduação em Saúde Coletiva}
    %\Faculdade{Faculdade de Filosofia Dom Aureliano Matos}
    \Centro{Centro de Ciências e Tecnologia}

    % Estes dois são mutuamente exclusivos.
    %\Especializacao{Especialização em Gestão Pedagógica da Escola Básica}
    \Curso{Curso de Graduação em Informática Licenciatura a Distância}

    \Cidade{Cidade --- Ceará}
    \AnoDeEntrega{Ano de Entrega}
    \Titulo{Um título de exemplo}
    \Subtitulo{um subtítulo}
    \TipoDeTrabalho{Dissertação}
    \DescricaoDoTrabalho{\xTipoDeTrabalho\ apresentada a \xUniversidade\
                         como cumprimento às exigências legais para obtenção do
                         título de Mestre.}
    \AreaDeConcentracao{Matemática}
    \Orientador{Dr.\,Nome}
    \Coorientador{Dr.\,Nome}
    \AvaliadorA{
        Dr.\,Nome \\
        Faculdade \\
        Universidade
    }
    \AvaliadorB{
        Dr.\,Nome \\
        Faculdade \\
        Universidade
    }
    \AvaliadorC{
        Dr.\,Nome \\
        Faculdade \\
        Universidade
    }
    \Dedicatoria{Dedico este trabalho a\dots}
    \Epigrafe{Linha de texto}
    \Resumo{brazil}{Parágrafo do Resumo.}
    \Resumo{english}{Paragraph.}
    \PalavrasChave{brazil}{
        Palavra 1,
        Palavra 2,
        Palavra 3,
        Palavra 4,
    }
    \PalavrasChave{english}{
        Keyword 1,
        Keyword 2,
        Keyword 3,
        Keyword 4,
    }

    \setmainfont{XITS}[
        UprightFont    = *-Regular,
        BoldFont       = *-Bold,
        ItalicFont     = *-Italic,
        BoldItalicFont = *-BoldItalic,
        Extension      = .otf
    ]
    \setmathfont{XITSMath-Regular.otf}

    \begin{document}
        \ImprimirPretextual
        \nonum\notoc\section{Lista de Figuras}
        \makelof
        \nonum\notoc\section{Sumário}
        \maketoc
        \section{Introdução}
        \lipsum[1]
        \legend{figure}{Um título}\src{Elaboração própria.}
        \begin{place}
            \includegraphics[width=.3\linewidth]{example-image}
        \end{place}
        \subsection{Uma seção secundária}
        \subsubsection{Uma seção terciária}
        \section{Desenvolvimento}
        \section{Conclusão}
    \end{document}
\end{longlst}

% MACROS PARA O CÓDIGO-FONTE
% ================================================

\pdfstringdefDisableCommands{\def\marks#1#2{}}
\appendix{\marks0{}\marks1{}Código-fonte\label{ap:codigo-fonte}}

\makeatletter
\newcounter{codesection}
\newcounter{codesubsection}[codesection]
\let\c@section=\c@codesection
\let\c@subsection=\c@codesubsection
\def\thecodesection{\arabic{codesection}}
\def\thecodesubsection{\thecodesection.\arabic{codesubsection}}
\edef\toclevel@codesection{\mainseclevel}
\edef\toclevel@codesubsection{\the\numexpr\mainseclevel-1\relax}

\appto\hooktocsection{\def\extleaders{}%
    \edef\savedskips{\rightskip=\the\rightskip \parfillskip=\the\parfillskip\relax}}
\appto\hooktocsubsection{\def\extleaders{}%
    \edef\savedskips{\rightskip=\the\rightskip \parfillskip=\the\parfillskip\relax}}
\appto\hookextline{\savedskips}

\def\toclabelbox{\lowercase{\eqbox{codetoc}}}
\def\tocsectionfont{\itshape\MakeUppercase}
\def\tocsubsectionfont{}
\def\sectionfont{\tocsectionfont}
\def\subsectionfont{\itshape}

% O \addcontentsline{#1}{#2}{#3} está programado para
% enviar marcadores apenas se #1 = toc. Podemos
% mudar isso com \hypersetup{bookmarkstype=<outro>}.
\hypersetup{bookmarkstype=codetoc}
\patchcmd{\targetsection}{\refstepcounter{section}}{\refstepcounter{codesection}}{}{}
\patchcmd{\targetsection}{\addcontentsline{toc}}{\addcontentsline{codetoc}}{}{}
\patchcmd{\targetsubsection}{\refstepcounter{subsection}}{\refstepcounter{codesubsection}}{}{}
\patchcmd{\targetsubsection}{\addcontentsline{toc}}{\addcontentsline{codetoc}}{}{}

\def\hooksection{}
\def\codesection{\section}
\def\codesubsection{\subsection}

\colorlet{C}{.}
\definecolor{tmp}{RGB}{128, 0, 32}    \colorlet{P}{tmp}
\definecolor{tmp}{RGB}{255, 228, 236} \colorlet{B}{tmp}

\expanded{\noexpand\lstdefinelanguage{mylang}{
    language    = [latex]tex,
    texcs       = {\directlua{for k,v in pairs(tex.primitives())
                   do tex.print(k, v.. ',')end}},
    deletetexcs = {end}
}}

\lstset{
    language    = mylang,
    frame       = single,
    numbers     = left,
    numbersep   = \dimexpr-5pt+\addnumbersep\relax,
    numberstyle = \fontsize{8bp}{9bp}\selectfont,
    literate    = {\^^M\%.}{}1,
}
\def\addnumbersep{9pt}
\mylstset{\linewidth}{0pt}{0pt}{3pt}{0pt}{3pt}{13pt+\addnumbersep}{0pt}{0pt}

\newcount\codecount \codecount=1
\newbox\mybox
\setbox\mybox=\hbox to\hsize{\hfil\abntsmall\ttfamily\bfseries\color{gray}abntexto-uece.cls}
\lstnewenvironment{code}[1][]{%
    \lstset{
        title       = \copy\mybox\nobreak,
        firstnumber = \codecount,
        #1
    }%
}{%
    \global\codecount=\value{lstnumber}\relax
}
\appto\hookXP{
    frame     = none,
    numbers   = none,
    numbersep = 0pt,
    gobble    = 4,
}
\parskip=0pt plus.2ex minus.1ex
\@starttoc{codetoc}
{\def\excl{!}\catcode\1\!=14 \catcode\1\%=9 %! abntexto-uece.cls
%! Public Domain Software 2025
%! 2025-08-27
%! 1.0
%! Elayson Abreu
%! abntexto.classe@gmail.com

% \codesection{Identificação da classe}

% \begin{code}
\NeedsTeXFormat{LaTeX2e}

\ProvidesClass{abntexto-uece}%
    [2025-08-23 1.0 Preparation of works in UECE standards]

\DeclareOption*{\PassOptionsToClass{\CurrentOption}{abntexto}}

\ProcessOptions

\LoadClass{abntexto}
%. \end{code}

% \codesection{Macros auxiliares}

% O comando `\RecortarEsquerda` remove espaço à esquerda a nível de processador de expansão, significando dizer que podemos fazer algo do tipo:
%\begin{xp}
%   \edef\y{ 123}
%   \edef\x{\RecortarEsquerda \y}
%   \show\x % >123.
%\end{xp}
% `\RecortarDireita` é análogo ao comando anterior e permite remover espaços à direita:
%\begin{xp}
%   \edef\y{123 }
%   \edef\x{\expandafter\RecortarDireita\expandafter{\y}}
%   \show\x % >123.
%\end{xp}
% `\RecortarAmbos` remove espaços à esquerda e direita.

% \begin{code}
\def\RecortarEsquerda{\romannumeral-`\.}

\count2=\catcode`?
\catcode`?=3
\def\RecortarDireita#1{\RecortarDireitaA#1? ?}
\def\RecortarDireitaA#1 ?{\RecortarDireitaB#1?}
\def\RecortarDireitaB#1?#2{#1}
\catcode`?=\count2

\def\RecortarAmbos#1{\expandafter\RecortarDireita\expandafter{\romannumeral-`\.#1}}
%. \end{code}

% \codesection{Macros de identificação do documento}

% Definindo macros de identificação do documento.

% \begin{code}
\def\Autor{\def\xAutor}
\def\Universidade{\def\xUniversidade}
\def\Centro{\def\xFaculdade{}\def\xPrograma{}\def\xCentro}
\def\Faculdade{\def\xCentro{}\def\xPrograma{}\def\xFaculdade}
\def\Programa{\def\xCentro{}\def\xFaculdade{}\def\xPrograma}
\def\Curso{\def\xEspecializacao{}\def\xCurso}
\def\Especializacao{\def\xCurso{}\def\xEspecializacao}
\def\SiglaDaInstituicao{\def\xSiglaDaInstituicao}
\def\Titulo{\def\xTitulo}
\def\Subtitulo{\def\xSubtitulo}
\def\Cidade{\def\xCidade}
\def\AnoDeEntrega{\def\xAnoDeEntrega}
\def\TipoDeTrabalho{\def\xTipoDeTrabalho}
\def\DescricaoDoTrabalho{\def\xDescricaoDoTrabalho}
\def\AreaDeConcentracao{\def\xAreaDeConcentracao}
\def\Orientador{\def\xOrientador}
\def\Coorientador{\def\xCoorientador}
\def\AvaliadorA{\def\xAvaliadorA}
\def\AvaliadorB{\def\xAvaliadorB}
\def\AvaliadorC{\def\xAvaliadorC}
\def\AvaliadorD{\def\xAvaliadorD}
\def\Agradecimentos{\def\xAgradecimentos}
\def\Dedicatoria{\def\xDedicatoria}
\def\Epigrafe{\def\xEpigrafe}
%. \end{code}

% `\xResumo` e `\xPalavrasChave` são definidos apenas como indicativo de que `\Resumo` e `\PalavrasChave` foram chamados pelo estudante e, adicionalmente, aqueles serão usados em `\CamposMandatorios`.

% `\Mandatorio\<atributo>` verifica se `\x<atributo>` está vazio ou indefinido. Se sim, uma mensagem de alerta é emitida pedindo que `\<atributo>{<valor>}` seja chamado no preâmbulo. `\CamposMandatorios` será chamado em `\begin{document}`.

% `\Centralizar` e `\Direita` serão úteis para alinhamento de caixas ou parágrafos de linhas únicas, por exemplo, `\Centralizar\begin{minipage}{<largura>}\end{minipage}`.

% \begin{code}
\def\Resumo#1#2{\def\xResumo{#2}\csdef{resumo/#1}{#2}}
\def\PalavrasChave#1#2{\def\xPalavrasChave{#2}\csdef{pchave/#1}{#2}%
    \def\tmp{#1}\@tempcnta=1 \doPalavrasChave#2,\relax,%
}
\def\doPalavrasChave#1,{\if&\RecortarEsquerda#1&\expandafter\doPalavrasChave \else
    \ifx\relax#1\else
        \csedef{xPalavraChave\@Alph\@tempcnta/\tmp}{\RecortarAmbos{#1}}%
        \advance\@tempcnta by1
        \expandafter\expandafter\expandafter \doPalavrasChave
    \fi\fi
}

\def\Mandatorio#1{\ifcsname x\expandafter\@gobble\string#1\endcsname
        \expandafter\ifx\csname x\expandafter\@gobble\string#1\endcsname\empty
        \ClassWarning{abntexto-uece}{Campo obrigatório vazio! "\string#1"}
        \csdef{x\expandafter\@gobble\string#1}{??}\fi
    \else
        \ClassWarning{abntexto-uece}{Campo obrigatório vazio! "\string#1"}
        \csdef{x\expandafter\@gobble\string#1}{??}
    \fi
}
\def\CamposMandatorios{
    \Mandatorio\Autor
    \Mandatorio\Universidade
    \Mandatorio\SiglaDaInstituicao
    \ifdefined\xCentro    \else
    \ifdefined\xFaculdade \else
    \ifdefined\xPrograma  \else
        \ClassWarning{abntexto-uece}{^^JUm dos três campos precisa ser chamado:
            \noexpand\Centro ou \noexpand\Faculdade ou \string\Programa!}
        \Centro{}\Faculdade{}\Programa{}%
    \fi\fi\fi
        \ifdefined\xCurso          \else
        \ifdefined\xEspecializacao \else
            \ClassWarning{abntexto-uece}{^^JUm dos dois campos precisa ser chamado:
                \noexpand\Curso ou \string\Especializacao!}
            \Curso{}\Especializacao{}
        \fi\fi
    \Mandatorio\SiglaDaInstituicao
    \Mandatorio\Titulo
    \Mandatorio\Subtitulo
    \Mandatorio\Cidade
    \Mandatorio\AnoDeEntrega
    \Mandatorio\TipoDeTrabalho
    \Mandatorio\DescricaoDoTrabalho
    \Mandatorio\AreaDeConcentracao
    \Mandatorio\Orientador
    \Mandatorio\Coorientador
    \Mandatorio\AvaliadorA
    \Mandatorio\Resumo
    \Mandatorio\PalavrasChave
}

\AtBeginDocument{\CamposMandatorios}

\def\Centralizar{\noindent\hfil}
\def\Direita{\noindent\hfill}
%. \end{code}

% \codesection{Impressão dos elementos pré-textuais}

% Definições de `\ImprimirCapa`, `\ImprimirFolhaDeRosto`, `\ImprimirFichaCatalografica`, `\ImprimirErrata`, `\ImprimirFolhaDeAprovacao`, `\ImprimirAvaliadores`, `\ImprimirDedicatoria`, `\ImprimirAgradecimentos`, `\ImprimirResumo`, `\ImprimirPalavrasChave`, `\ImprimirPretextual`, `\Metadados`. Todas essas intruções podem ser redefinidas pelo usuário.

% \begin{code}
\def\ImprimirCapa{
    \Centralizar
    \begin{minipage}{.7\linewidth}
        \centering
        \includegraphics[height=4\baselineskip]{example-image}\\
        \MakeUppercase{\xUniversidade}\\
        \MakeUppercase{\xCentro\xFaculdade\xPrograma}\\
        \MakeUppercase{\xCurso\xEspecializacao}\\
    \end{minipage}
    \Enter[3]
    \Centralizar\MakeUppercase{\xAutor}
    \Enter[5]
    \Centralizar
    \begin{minipage}{.7\linewidth}
        \centering\MakeUppercase{\xTitulo}
    \end{minipage}
    \vfill
    \Centralizar\MakeUppercase{\xCidade} % A próxima linha em branco é necessária.

    \Centralizar\MakeUppercase{\xAnoDeEntrega}
}
\def\ImprimirFolhaDeRosto{
    \Centralizar\MakeUppercase{\xAutor}
    \Enter[9]
    \Centralizar
    \begin{minipage}{.7\linewidth}
        \centering\MakeUppercase{\xTitulo}
    \end{minipage}
    \Enter[2]
    \Direita
    \begin{minipage}{.5\linewidth}
        \singlesp\nohyph
        \xDescricaoDoTrabalho
        \Enter
        Área de concentração: \xAreaDeConcentracao.\\
        Orientador(a): \xOrientador.\\
        Coorientador(a): \xCoorientador.
    \end{minipage}
    \vfill
    \Centralizar\MakeUppercase{\xCidade} % A próxima linha em branco é necessária.

    \Centralizar\MakeUppercase{\xAnoDeEntrega}
}
%. \end{code}

% É provável que a biblioteca de sua instituição forneça a ficha catalográfica em formato |.pdf|. Nesse caso, `\ImprimirFichaCatalografica` servirá apenas de preenchimento.

% \begin{code}
\def\CDU{000}
\def\CDD{A000}
\def\NumeroDeFolhas{100}
\edef\icardHeight{\the\dimexpr\icardHeight+4\baselineskip}

\def\ImprimirFichaCatalografica{
    \leavevmode\vfill
    \Centralizar Dados Internacionais de Catalogação na Publicação (CIP)
    \Enter[.5]
    \Centralizar
    \begin{indexcard}
        \noindent \CDD\hskip\parindent \qquad \xAutor

        \setbox0=\hbox{\CDD\qquad}\leftskip=\wd0 % A linha em branco antes
                                                 % dessa instrução é necessária.
        \xTitulo: \xSubtitulo\ / \xAutor\ --- \xCidade: \xUniversidade,
        \xSiglaDaInstituicao, \xAnoDeEntrega.

        \NumeroDeFolhas\,f.

        \xTipoDeTrabalho\ (\MakeUppercase{\xAreaDeConcentracao}) --- \xUniversidade,
        \xSiglaDaInstituicao: \xCidade, \xAnoDeEntrega.

        Orientador(a): \xOrientador.

        Coorientador(a): \xCoorientador.

        \ImprimirPalavrasChaveFicha{brazil}
        I. \xTitulo: \xSubtitulo.

        \Direita CDU \CDU
    \end{indexcard}
}
\def\ImprimirPalavrasChaveFicha#1{\def\tmp{#1}\@tempcnta=1 \doImprimirPalavrasChaveFicha}
\def\doImprimirPalavrasChaveFicha{%
    \ifcsname xPalavraChave\@Alph\@tempcnta/\tmp\endcsname
        \the\@tempcnta.~\csname xPalavraChave\@Alph\@tempcnta/\tmp\endcsname .\space
        \advance\@tempcnta by1
        \expandafter\doImprimirPalavrasChaveFicha
    \else
        \ignorespaces
    \fi
}
\def\ImprimirErrata{
    \nonum\notoc\section{Errata}
    \begingroup
        \ifdefined\fullcite \bibfont\fullcite{ferrigno2011}.\else
            \ClassWarning{abntexto-uece}{"biblatex.sty" ausente.}
        \fi
    \par\endgroup % Esse \par é necessário.
    \Enter
    \Centralizar
    \begin{tabular}{cccc}
        \hline
        \bfseries Folha & \bfseries Linha & \bfseries Onde se lê & \bfseries Leia-se \\ \hline
        16 & 10 & auto-clavado & autoclavado \\ \hline
    \end{tabular}
}
\def\ImprimirFolhaDeAprovacao{
    \Centralizar\xAutor
    \Enter[2]
    \Centralizar
    \begin{minipage}{.7\linewidth}
        \centering\xTitulo
    \end{minipage}
    \Enter[2]
    \Direita
    \begin{minipage}{.5\linewidth}
        \singlesp
        \xDescricaoDoTrabalho
        \Enter
        Área de concentração: \xAreaDeConcentracao.
    \end{minipage}
    \Enter
    {\parindent=1.5cm Aprovado em 00/00/0000.\par}
    \Enter
    \Centralizar BANCA EXAMINADORA
    \Enter
    \ImprimirAvaliadores
}
\def\ImprimirAvaliadores{\@tempcnta=1 \doImprimirAvaliadores}
\def\doImprimirAvaliadores{%
    \ifcsname xAvaliador\@Alph\@tempcnta\endcsname
        \judgeline{\csuse{xAvaliador\@Alph\@tempcnta}}%
        \Enter
        \advance\@tempcnta by1
        \expandafter\doImprimirAvaliadores
    \fi
}
\def\ImprimirDedicatoria{
    \leavevmode\vfill
    \Direita
    \begin{minipage}{\dimexpr\linewidth-8cm}
        \xDedicatoria
    \end{minipage}
}
\def\ImprimirAgradecimentos{
    \nonum\notoc\section{\bfseries Agradecimentos}
    {\parindent=1.5cm\xAgradecimentos\par}
}
\def\ImprimirEpigrafe{
    \leavevmode\vfil
    \Direita
    \begin{minipage}{5cm}
        \noindent \enquote{\xEpigrafe}\\ (autor,~ano)
    \end{minipage}
}
\def\ImprimirResumo#1#2{
    \nonum\notoc {\selectlanguage{#1}\section{\abstractname}}
    \noindent {\selectlanguage{#1}\csname resumo/#1\endcsname\par}
    \Enter
    \noindent \textbf{#2:}~%
    \begingroup \selectlanguage{#1}\nohyph
        \ImprimirPalavrasChave{#1}%
    \par\endgroup
}

\def\ImprimirPalavrasChave#1{%
    \expandafter\expandafter\expandafter \RecortarEsquerda
    \expandafter\@gobble
    \expanded{%
        \expandafter\expandafter\expandafter \ImprimirPalavrasChaveA
        \csname pchave/#1\endcsname ,\relax,%
    }%
}
\def\ImprimirPalavrasChaveA#1,{\if&\RecortarEsquerda#1&\expandafter\ImprimirPalavrasChaveA\else
    \ifx\relax#1.\else
        ;\space \RecortarAmbos{#1}%
        \expandafter\expandafter\expandafter \ImprimirPalavrasChaveA
    \fi\fi
}

\AtBeginDocument{\@ifpackageloaded{babel}{\relax}{\def\selectlanguage#1{}}}

\def\ImprimirPretextual{
    \ImprimirCapa               \newpage
    \ImprimirFolhaDeRosto       \newpage
    \ImprimirFichaCatalografica \newpage
    \ImprimirErrata             \newpage
    \ImprimirFolhaDeAprovacao   \newpage
    \ifdefined\xAgradecimentos \ImprimirAgradecimentos \fi \newpage
    \ifdefined\xDedicatoria    \ImprimirDedicatoria    \fi \newpage
    \ifdefined\xEpigrafe       \ImprimirEpigrafe       \fi \newpage
    \ImprimirResumo{brazil}{Palavras-chave} \newpage
    \ImprimirResumo{english}{Keywords}      \newpage
}
%. \end{code}

% \codesection{Metadados}

% \begin{code}
\def\Metadados{%
    \ifdefined\hypersetup \expandafter\hypersetup \else \expandafter\@gobble \fi
    {
        pdftitle    = \xTitulo,
        pdfauthor   = \xAutor,
        pdfcreator  = LaTeX with abntexto-uece,
        pdfkeywords = \ImprimirPalavrasChave{brazil},
    }%
}
\AtBeginDocument{\Metadados}
%. \end{code}

% \codesection{Ajustes finais}

% Trocando a palavra \enquote{Ilustração} por \enquote{Figura}. Essa redefinição não é necessária se você estiver usando o `abntexto` numa versão igual ou superior a 4.0.4-beta.

% \begin{code}
\definelegendplace{figure}{Figura}{lof}
%. \end{code}

% Indentação deixou de ser 1,5~cm e passou a ser 2~cm. O título das figuras/tabelas deve ser em negrito. \LaTeX\ emite um erro quando `\\` é chamado no modo vertical: o removeremos para que os alertas de `\Mandatorio` sejam destacados no terminal.

% \begin{code}
\parindent=2cm
\appto\targetlegend{\bfseries}
\let\@nolnerr=\relax

\endinput
%. \end{code}}
\end{document}